% Generated by Sphinx.
\def\sphinxdocclass{report}
\documentclass[letterpaper,10pt,english]{sphinxmanual}
\usepackage[utf8]{inputenc}
\DeclareUnicodeCharacter{00A0}{\nobreakspace}
\usepackage{cmap}
\usepackage[T1]{fontenc}
\usepackage{babel}
\usepackage{times}
\usepackage[Bjarne]{fncychap}
\usepackage{longtable}
\usepackage{sphinx}
\usepackage{multirow}

\addto\captionsenglish{\renewcommand{\figurename}{Fig. }}
\addto\captionsenglish{\renewcommand{\tablename}{Table }}
\floatname{literal-block}{Listing }



\title{python\_kobe Documentation}
\date{June 13, 2015}
\release{1.0}
\author{akira}
\newcommand{\sphinxlogo}{}
\renewcommand{\releasename}{Release}
\makeindex

\makeatletter
\def\PYG@reset{\let\PYG@it=\relax \let\PYG@bf=\relax%
    \let\PYG@ul=\relax \let\PYG@tc=\relax%
    \let\PYG@bc=\relax \let\PYG@ff=\relax}
\def\PYG@tok#1{\csname PYG@tok@#1\endcsname}
\def\PYG@toks#1+{\ifx\relax#1\empty\else%
    \PYG@tok{#1}\expandafter\PYG@toks\fi}
\def\PYG@do#1{\PYG@bc{\PYG@tc{\PYG@ul{%
    \PYG@it{\PYG@bf{\PYG@ff{#1}}}}}}}
\def\PYG#1#2{\PYG@reset\PYG@toks#1+\relax+\PYG@do{#2}}

\expandafter\def\csname PYG@tok@gd\endcsname{\def\PYG@tc##1{\textcolor[rgb]{0.63,0.00,0.00}{##1}}}
\expandafter\def\csname PYG@tok@gu\endcsname{\let\PYG@bf=\textbf\def\PYG@tc##1{\textcolor[rgb]{0.50,0.00,0.50}{##1}}}
\expandafter\def\csname PYG@tok@gt\endcsname{\def\PYG@tc##1{\textcolor[rgb]{0.00,0.27,0.87}{##1}}}
\expandafter\def\csname PYG@tok@gs\endcsname{\let\PYG@bf=\textbf}
\expandafter\def\csname PYG@tok@gr\endcsname{\def\PYG@tc##1{\textcolor[rgb]{1.00,0.00,0.00}{##1}}}
\expandafter\def\csname PYG@tok@cm\endcsname{\let\PYG@it=\textit\def\PYG@tc##1{\textcolor[rgb]{0.25,0.50,0.56}{##1}}}
\expandafter\def\csname PYG@tok@vg\endcsname{\def\PYG@tc##1{\textcolor[rgb]{0.73,0.38,0.84}{##1}}}
\expandafter\def\csname PYG@tok@m\endcsname{\def\PYG@tc##1{\textcolor[rgb]{0.13,0.50,0.31}{##1}}}
\expandafter\def\csname PYG@tok@mh\endcsname{\def\PYG@tc##1{\textcolor[rgb]{0.13,0.50,0.31}{##1}}}
\expandafter\def\csname PYG@tok@cs\endcsname{\def\PYG@tc##1{\textcolor[rgb]{0.25,0.50,0.56}{##1}}\def\PYG@bc##1{\setlength{\fboxsep}{0pt}\colorbox[rgb]{1.00,0.94,0.94}{\strut ##1}}}
\expandafter\def\csname PYG@tok@ge\endcsname{\let\PYG@it=\textit}
\expandafter\def\csname PYG@tok@vc\endcsname{\def\PYG@tc##1{\textcolor[rgb]{0.73,0.38,0.84}{##1}}}
\expandafter\def\csname PYG@tok@il\endcsname{\def\PYG@tc##1{\textcolor[rgb]{0.13,0.50,0.31}{##1}}}
\expandafter\def\csname PYG@tok@go\endcsname{\def\PYG@tc##1{\textcolor[rgb]{0.20,0.20,0.20}{##1}}}
\expandafter\def\csname PYG@tok@cp\endcsname{\def\PYG@tc##1{\textcolor[rgb]{0.00,0.44,0.13}{##1}}}
\expandafter\def\csname PYG@tok@gi\endcsname{\def\PYG@tc##1{\textcolor[rgb]{0.00,0.63,0.00}{##1}}}
\expandafter\def\csname PYG@tok@gh\endcsname{\let\PYG@bf=\textbf\def\PYG@tc##1{\textcolor[rgb]{0.00,0.00,0.50}{##1}}}
\expandafter\def\csname PYG@tok@ni\endcsname{\let\PYG@bf=\textbf\def\PYG@tc##1{\textcolor[rgb]{0.84,0.33,0.22}{##1}}}
\expandafter\def\csname PYG@tok@nl\endcsname{\let\PYG@bf=\textbf\def\PYG@tc##1{\textcolor[rgb]{0.00,0.13,0.44}{##1}}}
\expandafter\def\csname PYG@tok@nn\endcsname{\let\PYG@bf=\textbf\def\PYG@tc##1{\textcolor[rgb]{0.05,0.52,0.71}{##1}}}
\expandafter\def\csname PYG@tok@no\endcsname{\def\PYG@tc##1{\textcolor[rgb]{0.38,0.68,0.84}{##1}}}
\expandafter\def\csname PYG@tok@na\endcsname{\def\PYG@tc##1{\textcolor[rgb]{0.25,0.44,0.63}{##1}}}
\expandafter\def\csname PYG@tok@nb\endcsname{\def\PYG@tc##1{\textcolor[rgb]{0.00,0.44,0.13}{##1}}}
\expandafter\def\csname PYG@tok@nc\endcsname{\let\PYG@bf=\textbf\def\PYG@tc##1{\textcolor[rgb]{0.05,0.52,0.71}{##1}}}
\expandafter\def\csname PYG@tok@nd\endcsname{\let\PYG@bf=\textbf\def\PYG@tc##1{\textcolor[rgb]{0.33,0.33,0.33}{##1}}}
\expandafter\def\csname PYG@tok@ne\endcsname{\def\PYG@tc##1{\textcolor[rgb]{0.00,0.44,0.13}{##1}}}
\expandafter\def\csname PYG@tok@nf\endcsname{\def\PYG@tc##1{\textcolor[rgb]{0.02,0.16,0.49}{##1}}}
\expandafter\def\csname PYG@tok@si\endcsname{\let\PYG@it=\textit\def\PYG@tc##1{\textcolor[rgb]{0.44,0.63,0.82}{##1}}}
\expandafter\def\csname PYG@tok@s2\endcsname{\def\PYG@tc##1{\textcolor[rgb]{0.25,0.44,0.63}{##1}}}
\expandafter\def\csname PYG@tok@vi\endcsname{\def\PYG@tc##1{\textcolor[rgb]{0.73,0.38,0.84}{##1}}}
\expandafter\def\csname PYG@tok@nt\endcsname{\let\PYG@bf=\textbf\def\PYG@tc##1{\textcolor[rgb]{0.02,0.16,0.45}{##1}}}
\expandafter\def\csname PYG@tok@nv\endcsname{\def\PYG@tc##1{\textcolor[rgb]{0.73,0.38,0.84}{##1}}}
\expandafter\def\csname PYG@tok@s1\endcsname{\def\PYG@tc##1{\textcolor[rgb]{0.25,0.44,0.63}{##1}}}
\expandafter\def\csname PYG@tok@gp\endcsname{\let\PYG@bf=\textbf\def\PYG@tc##1{\textcolor[rgb]{0.78,0.36,0.04}{##1}}}
\expandafter\def\csname PYG@tok@sh\endcsname{\def\PYG@tc##1{\textcolor[rgb]{0.25,0.44,0.63}{##1}}}
\expandafter\def\csname PYG@tok@ow\endcsname{\let\PYG@bf=\textbf\def\PYG@tc##1{\textcolor[rgb]{0.00,0.44,0.13}{##1}}}
\expandafter\def\csname PYG@tok@sx\endcsname{\def\PYG@tc##1{\textcolor[rgb]{0.78,0.36,0.04}{##1}}}
\expandafter\def\csname PYG@tok@bp\endcsname{\def\PYG@tc##1{\textcolor[rgb]{0.00,0.44,0.13}{##1}}}
\expandafter\def\csname PYG@tok@c1\endcsname{\let\PYG@it=\textit\def\PYG@tc##1{\textcolor[rgb]{0.25,0.50,0.56}{##1}}}
\expandafter\def\csname PYG@tok@kc\endcsname{\let\PYG@bf=\textbf\def\PYG@tc##1{\textcolor[rgb]{0.00,0.44,0.13}{##1}}}
\expandafter\def\csname PYG@tok@c\endcsname{\let\PYG@it=\textit\def\PYG@tc##1{\textcolor[rgb]{0.25,0.50,0.56}{##1}}}
\expandafter\def\csname PYG@tok@mf\endcsname{\def\PYG@tc##1{\textcolor[rgb]{0.13,0.50,0.31}{##1}}}
\expandafter\def\csname PYG@tok@err\endcsname{\def\PYG@bc##1{\setlength{\fboxsep}{0pt}\fcolorbox[rgb]{1.00,0.00,0.00}{1,1,1}{\strut ##1}}}
\expandafter\def\csname PYG@tok@mb\endcsname{\def\PYG@tc##1{\textcolor[rgb]{0.13,0.50,0.31}{##1}}}
\expandafter\def\csname PYG@tok@ss\endcsname{\def\PYG@tc##1{\textcolor[rgb]{0.32,0.47,0.09}{##1}}}
\expandafter\def\csname PYG@tok@sr\endcsname{\def\PYG@tc##1{\textcolor[rgb]{0.14,0.33,0.53}{##1}}}
\expandafter\def\csname PYG@tok@mo\endcsname{\def\PYG@tc##1{\textcolor[rgb]{0.13,0.50,0.31}{##1}}}
\expandafter\def\csname PYG@tok@kd\endcsname{\let\PYG@bf=\textbf\def\PYG@tc##1{\textcolor[rgb]{0.00,0.44,0.13}{##1}}}
\expandafter\def\csname PYG@tok@mi\endcsname{\def\PYG@tc##1{\textcolor[rgb]{0.13,0.50,0.31}{##1}}}
\expandafter\def\csname PYG@tok@kn\endcsname{\let\PYG@bf=\textbf\def\PYG@tc##1{\textcolor[rgb]{0.00,0.44,0.13}{##1}}}
\expandafter\def\csname PYG@tok@o\endcsname{\def\PYG@tc##1{\textcolor[rgb]{0.40,0.40,0.40}{##1}}}
\expandafter\def\csname PYG@tok@kr\endcsname{\let\PYG@bf=\textbf\def\PYG@tc##1{\textcolor[rgb]{0.00,0.44,0.13}{##1}}}
\expandafter\def\csname PYG@tok@s\endcsname{\def\PYG@tc##1{\textcolor[rgb]{0.25,0.44,0.63}{##1}}}
\expandafter\def\csname PYG@tok@kp\endcsname{\def\PYG@tc##1{\textcolor[rgb]{0.00,0.44,0.13}{##1}}}
\expandafter\def\csname PYG@tok@w\endcsname{\def\PYG@tc##1{\textcolor[rgb]{0.73,0.73,0.73}{##1}}}
\expandafter\def\csname PYG@tok@kt\endcsname{\def\PYG@tc##1{\textcolor[rgb]{0.56,0.13,0.00}{##1}}}
\expandafter\def\csname PYG@tok@sc\endcsname{\def\PYG@tc##1{\textcolor[rgb]{0.25,0.44,0.63}{##1}}}
\expandafter\def\csname PYG@tok@sb\endcsname{\def\PYG@tc##1{\textcolor[rgb]{0.25,0.44,0.63}{##1}}}
\expandafter\def\csname PYG@tok@k\endcsname{\let\PYG@bf=\textbf\def\PYG@tc##1{\textcolor[rgb]{0.00,0.44,0.13}{##1}}}
\expandafter\def\csname PYG@tok@se\endcsname{\let\PYG@bf=\textbf\def\PYG@tc##1{\textcolor[rgb]{0.25,0.44,0.63}{##1}}}
\expandafter\def\csname PYG@tok@sd\endcsname{\let\PYG@it=\textit\def\PYG@tc##1{\textcolor[rgb]{0.25,0.44,0.63}{##1}}}

\def\PYGZbs{\char`\\}
\def\PYGZus{\char`\_}
\def\PYGZob{\char`\{}
\def\PYGZcb{\char`\}}
\def\PYGZca{\char`\^}
\def\PYGZam{\char`\&}
\def\PYGZlt{\char`\<}
\def\PYGZgt{\char`\>}
\def\PYGZsh{\char`\#}
\def\PYGZpc{\char`\%}
\def\PYGZdl{\char`\$}
\def\PYGZhy{\char`\-}
\def\PYGZsq{\char`\'}
\def\PYGZdq{\char`\"}
\def\PYGZti{\char`\~}
% for compatibility with earlier versions
\def\PYGZat{@}
\def\PYGZlb{[}
\def\PYGZrb{]}
\makeatother

\renewcommand\PYGZsq{\textquotesingle}

\begin{document}

\maketitle
\tableofcontents
\phantomsection\label{index::doc}


Contents:


\chapter{Gitの使い方について}
\label{chap1/sec1:python}\label{chap1/sec1:git}\label{chap1/sec1::doc}
GitHubでpullしたり,branchを作ってそれをローカルで再現する方法がややこしいのでまとめのメモを残しておきます.

そもそも,kenjisatoさんがこの作業の流れは書いてくださっているので,そちらを \href{https://github.com/Akira55/sphinx/pull/7}{参照} すると詳しいです.

ですが,このセクションでは,自分の確認も兼ねて,もう一度流れを確認します.

わかりやすいように,SphinxのGitHubで作業を行っている前提でチュートリアルを書きます.

master branchとは,親ファイルのことです.
この,masterの内容を修正したり,改善したいときは別に,branchを作ってそこに内容を書き加えます.branchを作った時点では,masterと同じ内容がコピーされます.

こうすることで,親ファイル(master)の内容をいじらずに,改善を加えた親のコピー(branch)を見せることができます.

branchをつくる方法は,:

\begin{Verbatim}[commandchars=\\\{\}]
\PYGZdl{}git branch brancのなまえ
\end{Verbatim}

で作ることができます.
ただし,branchを作った時点では,そのbranchに移動しているわけではないので,注意が必要です.

最終的には,branchは破棄されるか,masterにマージされるかするのですが.まずは,誰かが作った,branchの内容を自分のローカル環境で再現する方法を確認してみましょう.
\begin{itemize}
\item {} 
master以外のブランチに移動して,その内容を再現する方法.

\end{itemize}

まず{}`git status{}`でどこのブランチにいるか,そのブランチにおける(自分が行った)更新がどのようなものかの一覧を出して確認します.

これらの更新を保存したければ{}`git stash{}`を使って,保存できます.

この作業はmasterのbrachでなくても同じ.つまり.誰かのbranchに移って再現するときも同じ操作をする.

次に,:

\begin{Verbatim}[commandchars=\\\{\}]
\PYGZdl{}checkout branch名
\end{Verbatim}

で他のbranchにうつる.

すると,ローカルでそのbranchを再現できるようになる.つまり,見かけはそのbranchの内容が上書きされたように見える.

あれこれ確認できたら,:

\begin{Verbatim}[commandchars=\\\{\}]
\PYGZdl{}stash
\end{Verbatim}

をして保存(必要があれば)して,master branchに:

\begin{Verbatim}[commandchars=\\\{\}]
\PYGZdl{}checkout master
\end{Verbatim}

で戻る.

もし,作ったbranchをリモートブランチにpushしたければ:

\begin{Verbatim}[commandchars=\\\{\}]
\PYGZdl{}git push origin branchの名前
\end{Verbatim}

でpushする.Terminal(コマンド・プロンプト)でpushした場合でも,GitHubのユーザー名とパスワードを入力を要求される場合があるので,従う.


\chapter{pythonを勉強する周辺環境}
\label{chap1/sec2:python}\label{chap1/sec2::doc}
pythonの勉強会をするにあたって,周辺環境についても指導を頂きました.
とりあえず,書き並べて,それぞれ別の章で詳しい説明を行う予定です.

\textbf{Sphinx}

Pythonで書かれたテキストエディタです.
.rstファイルで,テキストを書くと,latexやhtmlでコンパイルしてくれる優れものです.

.rstファイルの記法はreST(reStructuredText)記法に基いています.
詳しくは, {\color{red}\bfseries{}公式のreStructuredText入門\_} か, \href{http://d.hatena.ne.jp/kk\_Ataka/20111202/1322839748}{Sphinxでドキュメントを書くためreST記法に入門した} を参照するといいと思います.
\begin{quote}
\end{quote}

(要議論)章立てについて

\textbf{テキストエディタについて}
\begin{itemize}
\item {} 
sublime text

\end{itemize}

を薦めて頂きました.使いやすそうです.
今後,sphinxはsublime text で編集してみようと思います.
Terminal から起動するには,pathを通す必要があるようです.
(僕のMacでは,sublがコマンドに割り当てられています.)

\textbf{Git Hub}

GitHubとはバージョン管理システムである,Gitを使うサイトです.
GitHubを使うことで,プログラムソースのバージョン管理が容易になります.

共同研究やプロジェクトを行うには,今後,GitHubを使えるようになることが必須だろうと予測されるので,是非習熟したいです.

この勉強会の教科書になっている,Quantitive EconomicのSolutionや,Libraryをつくるプロジェクトも \href{https://github.com/QuantEcon/QuantEcon.py}{GitHubで公開} されています.

そして,そもそもこの勉強会の資料なども,Githubで \href{https://github.com/Akira55/sphinx}{公開してます} .
\begin{quote}
\end{quote}

\textbf{JupyterノートブックをRSTに変換するプログラム}

kenjisatoさんが,Jupyter ノートブックを RSTに変換するスクリプト notebookconvert.py を作りました.

Makefile のあるディレクトリで:

\begin{Verbatim}[commandchars=\\\{\}]
\PYGZdl{} python notebookconvert.py
\PYGZdl{} make html
\end{Verbatim}

として出力すれば自動的に変換されます.


\chapter{これまでの取り決めの流れ}
\label{chap1/sec3::doc}\label{chap1/sec3:id1}
普通に,Quantitive\_Economic\_ の教材を薦めて行く予定だったのですが,プログラミングを学ぶ上でGitといった周辺環境を構築したり,基礎知識を学ぶ必要性を認識しました.

具体的な勉強会の取り決めは以下の通りです.
\begin{quote}
\begin{itemize}
\item {} 
プログラムの資料は基本的にJupyter Notebook(旧 python notebook)を使う

\item {} 
勉強会後に,新たに分かったことを書き加えた講義ノートをSphinxで作成

この理由として
\begin{quote}
\begin{enumerate}
\item {} 
jupyter Notebookのような形式ではなく.講義資料としてまとまったものを作成したいという要請

\item {} 
Githubを使う練習としては,SphinxのほうがJupiter notebook よりも適切であること

\end{enumerate}

の2点の理由が挙げられます.
\end{quote}

\item {} 
\end{itemize}
\end{quote}


\chapter{QUANTITATIVE ECONOMICS with Python}
\label{chap2/preface:quantitative-economics-with-python}\label{chap2/preface::doc}
QUANTITATIVE ECONOMICS with Python に基いて行った,勉強会の講義ノートです.

勉強会当日はJupiter Notebookで資料を作成します.

ですが,勉強会中に気付いたことや,もらった有益アドバイスなどをまとめるためにノートを作成します.

勉強会は教科書にしたがって進みますが,講義資料は随時更新されていきます.


\chapter{Python Essentials(p48)}
\label{chap2/python_essential/sec1::doc}\label{chap2/python_essential/sec1:python-essentials-p48}
\href{http://quant-econ.net/py/python\_essentials.html\#id3}{Python\_Essentials}
\begin{quote}
\end{quote}


\section{Data Type(p49)}
\label{chap2/python_essential/sec1:data-type-p49}
\href{http://quant-econ.net/py/python\_essentials.html\#id5}{Data\_Type} では,Pythonでよく使われるデータの種類について学びます.


\subsection{Primitive Data Types}
\label{chap2/python_essential/sec1:primitive-data-types}

\subsubsection{Bool Type}
\label{chap2/python_essential/sec1:bool-type}
最も基本的なデータタイプはBool型です.

これは,変数自体にTrueやFalseが入ります.:

\begin{Verbatim}[commandchars=\\\{\}]
\PYG{n}{x} \PYG{o}{=} \PYG{n+nb+bp}{True}
\end{Verbatim}

という感じです.

また,命題についても,True or Falseに変換されて変数に入ります.:

\begin{Verbatim}[commandchars=\\\{\}]
\PYG{n}{y} \PYG{o}{=} \PYG{l+m+mi}{100}\PYG{o}{\PYGZlt{}}\PYG{l+m+mi}{10}
\end{Verbatim}

のようにすると,yにはFalseが入ります.


\subsubsection{Data typeの確認方法}
\label{chap2/python_essential/sec1:data-type}
プログラムを書いていて,ある変数がどのようなデータタイプなのか,わからなくなることがあります.

これ以降にも,様々なデータタイプが出てきますが,先にデータタイプの確認方法を学びましょう.

ある,変数xがどのような    データタイプかを調べるためには:

\begin{Verbatim}[commandchars=\\\{\}]
\PYG{n+nb}{type}\PYG{p}{(}\PYG{n}{x}\PYG{p}{)}
\end{Verbatim}

とします.

もし,xが   Bool typeならば:

\begin{Verbatim}[commandchars=\\\{\}]
\PYG{n+nb}{bool}
\end{Verbatim}

と出力されます.


\subsubsection{Bool Typeの性質}
\label{chap2/python_essential/sec1:id1}
Bool Typeの変数にはTrueもしくはFalseが割り当てられますが,TrueとFalseにはそれぞれ,1と2という数字も割り当てられています.

もしxがTrue,yがFalseならば:

\begin{Verbatim}[commandchars=\\\{\}]
\PYG{n}{x}\PYG{o}{+}\PYG{n}{Y}

\PYG{l+m+mi}{1}
\end{Verbatim}

となります.


\subsubsection{ListとBool}
\label{chap2/python_essential/sec1:listbool}
この,Bool TypeはListの要素にもなります.


\subsubsection{Listとは}
\label{chap2/python_essential/sec1:list}
Listは様々なオブジェクトを格納できる列のことです.

Listには,値や文字なども入れることができます.例えば,:

\begin{Verbatim}[commandchars=\\\{\}]
\PYG{n}{bools} \PYG{o}{=} \PYG{p}{[}\PYG{n+nb+bp}{True}\PYG{p}{,} \PYG{n+nb+bp}{True}\PYG{p}{,} \PYG{n+nb+bp}{False}\PYG{p}{,} \PYG{n+nb+bp}{True}\PYG{p}{]}
\end{Verbatim}

という感じです.先ほどの,TrueやFalseに1と0が割り当てられることを考えると,:

\begin{Verbatim}[commandchars=\\\{\}]
\PYG{n+nb}{sum}\PYG{p}{(}\PYG{n}{bools}\PYG{p}{)}

\PYG{l+m+mi}{3}
\end{Verbatim}

となります.このsum()のような命令のことを関数と呼びます.

特に,このsum()は組み込み関数とよばれ,pythonにもともと入っています.

ほかの組み込み関数については,\href{http://docs.python.jp/2/library/functions.html\#sum}{公式の組み込み関数} を参照してください.


\subsubsection{そのほかのデータータイプ}
\label{chap2/python_essential/sec1:id2}
PythonにはBool Type以外にのデータタイプも存在します.

例えば数字の,integersとfloatsの2つの種類のデータタイプがあります.:

\begin{Verbatim}[commandchars=\\\{\}]
\PYG{n}{a}\PYG{p}{,} \PYG{n}{b} \PYG{o}{=} \PYG{l+m+mi}{1}\PYG{p}{,} \PYG{l+m+mi}{2}
\PYG{n+nb}{type}\PYG{p}{(}\PYG{n}{a}\PYG{p}{)}

\PYG{n+nb}{int}
\end{Verbatim}

がinterger Typeであり,:

\begin{Verbatim}[commandchars=\\\{\}]
\PYG{n}{c}\PYG{p}{,} \PYG{n}{d} \PYG{o}{=} \PYG{l+m+mf}{2.5}\PYG{p}{,} \PYG{l+m+mf}{10.0}
\PYG{n+nb}{type}\PYG{p}{(}\PYG{n}{c}\PYG{p}{)}

\PYG{n+nb}{float}
\end{Verbatim}

となります.このintとfloatについては後に詳しい説明をします.


\subsubsection{注意}
\label{chap2/python_essential/sec1:id3}
この,integerに関連する問題を一つ見てみましょう.

Python 2x では,2つのinteger(整数)同士の割り算では,integerの部分だけを返します.

つまり,:

\begin{Verbatim}[commandchars=\\\{\}]
\PYG{l+m+mi}{1}\PYG{o}{/}\PYG{l+m+mi}{2}

\PYG{l+m+mi}{0}
\end{Verbatim}

となります.

ただし,integerとfloatや,floatとfloat同士の割り算では,少数以下も返されます.:

\begin{Verbatim}[commandchars=\\\{\}]
\PYG{l+m+mf}{1.0}\PYG{o}{/}\PYG{l+m+mf}{2.0}

\PYG{l+m+mf}{0.5}
\end{Verbatim}

ですし,:

\begin{Verbatim}[commandchars=\\\{\}]
\PYG{l+m+mf}{1.0}\PYG{o}{/}\PYG{l+m+mi}{2}

\PYG{l+m+mf}{0.5}
\end{Verbatim}

となります.

また,このような問題はPython 3xでは発生しません.

しかし,この教科書はPyhton 2xを用いるので,読者はこのような問題に留意する必要があるでしょう.


\subsubsection{complex Type}
\label{chap2/python_essential/sec1:complex-type}
複素数も,PythonにおけるPrimitiveなデータタイプの一つです.

Pythonでは,Complex Type と呼ばれます.

Pythonで複素数を表現するには,組み込み関数のcomplex()を使います.:

\begin{Verbatim}[commandchars=\\\{\}]
complex(実部,虚部)
\end{Verbatim}

のように指定します.また,Pythonでは複素数はjで表現されます.:

\begin{Verbatim}[commandchars=\\\{\}]
\PYG{n}{x} \PYG{o}{=} \PYG{n+nb}{complex}\PYG{p}{(}\PYG{l+m+mi}{1}\PYG{p}{,} \PYG{l+m+mi}{2}\PYG{p}{)}
\PYG{n}{y} \PYG{o}{=} \PYG{n+nb}{complex}\PYG{p}{(}\PYG{l+m+mi}{2}\PYG{p}{,} \PYG{l+m+mi}{1}\PYG{p}{)}
\end{Verbatim}

とすれば,:

\begin{Verbatim}[commandchars=\\\{\}]
\PYG{n}{x}\PYG{o}{*}\PYG{n}{y}

\PYG{l+m+mi}{5j}
\end{Verbatim}

となります.


\subsubsection{Containers p50}
\label{chap2/python_essential/sec1:containers-p50}
Pythonには様々なコンテナが存在します.
コンテナは,データーを集めておくために使われます.

例えば,先に説明した,listは組み込みコンテナといって,Pythonにもともと備わっています.

listと同じような,組み込みコンテナとしてtuple(トゥープル,タプル)があります.

この,tupleとlistの大きな違いの一つに,tupuleがimmutableであることが挙げられます.

“tupleがimmutable”とは,tupleの値が変更できないことを意味します.

一方で,“listはmutable”なので,値を変更することができます.

以下に例を示します,まずlistはmutableすなわち持っている変数の数が増えたり,減ったり変わったりします.:

\begin{Verbatim}[commandchars=\\\{\}]
\PYG{n}{x} \PYG{o}{=} \PYG{p}{[}\PYG{l+m+mi}{1}\PYG{p}{,} \PYG{l+m+mi}{2}\PYG{p}{]}
\end{Verbatim}

というlistを考え,この1行目の,1を変化させてみましょう.

ところで,この行番号ですが,pythonでは,0から数えます.
x{[}0{]}というようにすると,listであるxの0行目を指定できます.これを変更するには,:

\begin{Verbatim}[commandchars=\\\{\}]
\PYG{n}{x}\PYG{p}{[}\PYG{l+m+mi}{0}\PYG{p}{]}\PYG{o}{=}\PYG{l+m+mi}{10}
\end{Verbatim}

というようにします.確認すると,:

\begin{Verbatim}[commandchars=\\\{\}]
\PYG{n}{x}

\PYG{p}{[}\PYG{l+m+mi}{10}\PYG{p}{,} \PYG{l+m+mi}{2}\PYG{p}{]}
\end{Verbatim}

というように,変更されていることがわかります.

このように,listはmutablです.しかし,一方で,tupuleはimmutableです.:

\begin{Verbatim}[commandchars=\\\{\}]
\PYG{n}{X} \PYG{o}{=} \PYG{p}{(}\PYG{l+m+mi}{1}\PYG{p}{,} \PYG{l+m+mi}{2}\PYG{p}{)}
\end{Verbatim}

にたいして,X{[}0{]}とすると,:

\begin{Verbatim}[commandchars=\\\{\}]
X[0] = 10

\PYGZhy{}\PYGZhy{}\PYGZhy{}\PYGZhy{}\PYGZhy{}\PYGZhy{}\PYGZhy{}\PYGZhy{}\PYGZhy{}\PYGZhy{}\PYGZhy{}\PYGZhy{}\PYGZhy{}\PYGZhy{}\PYGZhy{}\PYGZhy{}\PYGZhy{}\PYGZhy{}\PYGZhy{}\PYGZhy{}\PYGZhy{}\PYGZhy{}\PYGZhy{}\PYGZhy{}\PYGZhy{}\PYGZhy{}\PYGZhy{}\PYGZhy{}\PYGZhy{}\PYGZhy{}\PYGZhy{}\PYGZhy{}\PYGZhy{}\PYGZhy{}\PYGZhy{}\PYGZhy{}\PYGZhy{}\PYGZhy{}\PYGZhy{}\PYGZhy{}\PYGZhy{}\PYGZhy{}\PYGZhy{}\PYGZhy{}\PYGZhy{}\PYGZhy{}\PYGZhy{}\PYGZhy{}\PYGZhy{}\PYGZhy{}\PYGZhy{}\PYGZhy{}\PYGZhy{}\PYGZhy{}\PYGZhy{}\PYGZhy{}\PYGZhy{}\PYGZhy{}\PYGZhy{}\PYGZhy{}\PYGZhy{}\PYGZhy{}\PYGZhy{}\PYGZhy{}\PYGZhy{}\PYGZhy{}\PYGZhy{}\PYGZhy{}\PYGZhy{}\PYGZhy{}\PYGZhy{}\PYGZhy{}\PYGZhy{}\PYGZhy{}\PYGZhy{}
TypeError                                 Traceback (most recent call last)
\PYGZlt{}ipython\PYGZhy{}input\PYGZhy{}7\PYGZhy{}531149b57146\PYGZgt{} in \PYGZlt{}module\PYGZgt{}()
\PYGZhy{}\PYGZhy{}\PYGZhy{}\PYGZhy{}\PYGZgt{} 1 X[0] = 10

TypeError: \PYGZsq{}tuple\PYGZsq{} object does not support item assignment
\end{Verbatim}

となってしまいます


\subsubsection{もう少し,mutable とimmutableの話をしよう}
\label{chap2/python_essential/sec1:mutable-immutable}
mutableなlistにも,immutableなリストにも,''unpacked''という操作を施すことができます.

unpackedでは,それぞれの行を指定した変数に当てはめることができます.:

\begin{Verbatim}[commandchars=\\\{\}]
\PYG{n}{integers} \PYG{o}{=} \PYG{p}{(}\PYG{l+m+mi}{10}\PYG{p}{,} \PYG{l+m+mi}{20}\PYG{p}{,} \PYG{l+m+mi}{30}\PYG{p}{)}
\PYG{n}{x}\PYG{p}{,} \PYG{n}{y}\PYG{p}{,} \PYG{n}{z} \PYG{o}{=} \PYG{n}{integers}
\end{Verbatim}

とすると:

\begin{Verbatim}[commandchars=\\\{\}]
\PYG{n}{x}
\PYG{l+m+mi}{10}

\PYG{n}{y}
\PYG{l+m+mi}{20}

\PYG{n}{z}
\PYG{l+m+mi}{30}
\end{Verbatim}

というように割り当てられます.

また,slice notetion という操作も,mutable,immutableのどちらにも施すことができます.

例えば,:

\begin{Verbatim}[commandchars=\\\{\}]
\PYG{n}{a} \PYG{o}{=} \PYG{p}{[}\PYG{l+m+mi}{2}\PYG{p}{,} \PYG{l+m+mi}{4} \PYG{p}{,} \PYG{l+m+mi}{6}\PYG{p}{,} \PYG{l+m+mi}{8}\PYG{p}{]}
\end{Verbatim}

という,listの1行目から,最後の行までを抜き出したいときは,:

\begin{Verbatim}[commandchars=\\\{\}]
\PYG{n}{a}\PYG{p}{[}\PYG{l+m+mi}{1}\PYG{p}{:}\PYG{p}{]}
\PYG{p}{[}\PYG{l+m+mi}{4}\PYG{p}{,} \PYG{l+m+mi}{6}\PYG{p}{,} \PYG{l+m+mi}{8}\PYG{p}{]}
\end{Verbatim}

と指定します.

また,ある行から,ある行までを抜き出したいとき,例えば,1-2行目を抜き出したいとき,:

\begin{Verbatim}[commandchars=\\\{\}]
\PYG{n}{a}\PYG{p}{[}\PYG{l+m+mi}{1}\PYG{p}{:}\PYG{l+m+mi}{3}\PYG{p}{]}
\PYG{p}{[}\PYG{l+m+mi}{4}\PYG{p}{,} \PYG{l+m+mi}{6}\PYG{p}{]}
\end{Verbatim}

というような指定の仕方をします.:

\begin{Verbatim}[commandchars=\\\{\}]
list[抜き出しを開始する行番号:抜き出しを終わる行番号+1]
\end{Verbatim}

という感じです.

また,:

\begin{Verbatim}[commandchars=\\\{\}]
\PYG{n}{a}\PYG{p}{[}\PYG{o}{\PYGZhy{}}\PYG{l+m+mi}{2}\PYG{p}{:}\PYG{p}{]}
\PYG{p}{[}\PYG{l+m+mi}{6}\PYG{p}{,} \PYG{l+m+mi}{8}\PYG{p}{]}
\end{Verbatim}

というようにすれば,最後の2行を抜き出すことができます.

以上の一連の操作は,文字に対しても行えて,:

\begin{Verbatim}[commandchars=\\\{\}]
\PYG{n}{s} \PYG{o}{=} \PYG{l+s}{\PYGZsq{}}\PYG{l+s}{kobe univ.}\PYG{l+s}{\PYGZsq{}}

\PYG{n}{s}\PYG{p}{[}\PYG{o}{\PYGZhy{}}\PYG{l+m+mi}{5}\PYG{p}{:}\PYG{p}{]}
\PYG{l+s}{\PYGZsq{}}\PYG{l+s}{univ.}\PYG{l+s}{\PYGZsq{}}
\end{Verbatim}

と抜き出せることができます.

このような,最後の数行を抜き出すという操作は,全体を確認するには長すぎるデータの内容を確認するときに,有効な場合があります.


\subsubsection{Sets と Dictionaries}
\label{chap2/python_essential/sec1:sets-dictionaries}
先に,listとtupulという二種類のcontainerを紹介しました.

次に,setとdictionaryという2つのcontainerについて説明します.
\begin{quote}
\end{quote}



\renewcommand{\indexname}{Index}
\printindex
\end{document}
